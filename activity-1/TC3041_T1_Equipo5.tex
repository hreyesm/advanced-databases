%%
%% This is file `sample-acmsmall.tex',
%% generated with the docstrip utility.
%%
%% The original source files were:
%%
%% samples.dtx  (with options: `acmsmall')
%% 
%% IMPORTANT NOTICE:
%% 
%% For the copyright see the source file.
%% 
%% Any modified versions of this file must be renamed
%% with new filenames distinct from sample-acmsmall.tex.
%% 
%% For distribution of the original source see the terms
%% for copying and modification in the file samples.dtx.
%% 
%% This generated file may be distributed as long as the
%% original source files, as listed above, are part of the
%% same distribution. (The sources need not necessarily be
%% in the same archive or directory.)
%%
%% The first command in your LaTeX source must be the \documentclass command.
\documentclass[acmsmall]{acmart}

\usepackage{graphicx}  % for images

%%
%% \BibTeX command to typeset BibTeX logo in the docs
\AtBeginDocument{%
  \providecommand\BibTeX{{%
    \normalfont B\kern-0.5em{\scshape i\kern-0.25em b}\kern-0.8em\TeX}}}

%% Rights management information.  This information is sent to you
%% when you complete the rights form.  These commands have SAMPLE
%% values in them; it is your responsibility as an author to replace
%% the commands and values with those provided to you when you
%% complete the rights form.
\setcopyright{acmcopyright}
\copyrightyear{2021}
\acmYear{2021}

%%
%% These commands are for a JOURNAL article.
\acmJournal{JACM}
\acmMonth{1}

%%
%% Submission ID.
%% Use this when submitting an article to a sponsored event. You'll
%% receive a unique submission ID from the organizers
%% of the event, and this ID should be used as the parameter to this command.
%%\acmSubmissionID{123-A56-BU3}

%%
%% The majority of ACM publications use numbered citations and
%% references.  The command \citestyle{authoryear} switches to the
%% "author year" style.
%%
%% If you are preparing content for an event
%% sponsored by ACM SIGGRAPH, you must use the "author year" style of
%% citations and references.
%% Uncommenting
%% the next command will enable that style.
%%\citestyle{acmauthoryear}

%%
%% end of the preamble, start of the body of the document source.
\begin{document}

%%
%% The "title" command has an optional parameter,
%% allowing the author to define a "short title" to be used in page headers.
\title{Actividad 1. Bases de datos en memoria}

%%
%% The "author" command and its associated commands are used to define
%% the authors and their affiliations.
%% Of note is the shared affiliation of the first two authors, and the
%% "authornote" and "authornotemark" commands
%% used to denote shared contribution to the research.
\author{Daniela Vignau (A01021698)}
\affiliation{%
  \institution{Tecnológico de Monterrey}
  \country{México}}

\author{Cristopher Cejudo (A01021698)}
\affiliation{%
  \institution{Tecnológico de Monterrey}
  \country{México}}

\author{Héctor Reyes (A01339607)}
\affiliation{%
  \institution{Tecnológico de Monterrey}
  \country{México}}

%%
%% By default, the full list of authors will be used in the page
%% headers. Often, this list is too long, and will overlap
%% other information printed in the page headers. This command allows
%% the author to define a more concise list
%% of authors' names for this purpose.
\renewcommand{\shortauthors}{Daniela Vignau, Cristopher Cejudo, Héctor Reyes}

%%
%% The abstract is a short summary of the work to be presented in the
%% article.
\begin{abstract}
  Este reporte busca profundizar en varios aspectos de las bases de datos en memoria, teniendo como foco el sistema de administración de bases de datos {\itshape Kinetica}. También se analiza el rendimiento de la base de datos en cuestión para las operaciones de inserción y consulta.
\end{abstract}

%%
%% This command processes the author and affiliation and title
%% information and builds the first part of the formatted document.
\maketitle

\section{Introducción}
Kinetica es una base de datos distribuida en memoria que permite ingerir, analizar y visualizar datos simultáneamente. Es un sistema de administración de bases de datos (DBMS) único en el sentido de que aprovecha el rendimiento de la unidad de procesamiento gráfico (GPU) para realizar operaciones con más rapidez que las bases de datos tradicionales.

Al tratarse de una base de datos orientada a columnas diseñada para el procesamiento analítico (OLAP), Kinetica está optimizada para manejar grandes volúmenes de datos de alta cardinalidad. Por tanto, no es adecuada como sistema de uso transaccional (OLTP). Kinetica organiza los datos de forma estructurada, similar a otras bases de datos relacionales, y los almacena en la memoria RAM o vRAM, en el caso de las GPUs.

\section{Desarrollo}

\subsection{Arquitectura del DBMS}
\begin{figure}[h]
  \includegraphics[width=100mm, scale=0.5]{./images/dbms-architecture.jpg}
  \caption{Arquitectura del DBMS}
  \label{fig:Arquitectura del DBMS}
\end{figure}

Kinética cuenta con una arquitectura distribuida diseñada así para el procesamiento de datos a escala. Un clúster, compuesto por nodos idénticos, es capaz de correr en hardware básico así como también en aquellos equipados con una unidad de procesamiento de gráficos (GPU, por su siglas en inglés). Uno de esos nodos, es seleccionado para ser el nodo de agregación principal. En la figura \ref{fig:Arquitectura del DBMS} se muestra un diagrama sobre ella.


\subsection{Tipo de almacenamiento utilizado}
La interfaz API nativa al sistema para el almacenamiento de datos, es de tipo objetos, con cada uno de estos siendo una fila en la tabla.

\subsection{Representación en memoria}
Dado que el tipo de almacenamiento es por columnas, la representación en memoria es de la secuencial, todos los datos de una columna están después del otro, e inmediatamente después de terminar dicha columna, comienza la siguiente.

\subsection{Mecanismos de compresión}
{\itshape Nota: No estoy muy segura de si está bien esto}

La compresión puede ser aplicada a una columna de una tabla. Por default, la columna es almacenada sin compresión y una vez que se comprime, se mantiene en ese estado hasta que es usada. Cuando los datos son recuperados, se realiza una copia temporal de los datos descomprimidos y se descarta una vez que se deje de utilizar. Los mecanismos de compresión disponibles en Kinética no están claramente especificados en la documentación, por lo que se puede llegar asumir que se permite cualquier algoritmo.

\subsection{Particionamiento}
{\itshape Nota: ¿Será necesario escribir más al respecto?}

Los datos de una tabla pueden ser sometidos a un particionamiento para obtener mejoras en el rendimiento y en el manejo de los datos. Los esquemas de particionamiento permitidos son:
\begin{itemize}
  \item Rango
  \item Intervalo
  \item Lista
  \item Hash
  \item Series
\end{itemize}

\subsection{Operaciones DML}
\subsection{Buffer diferencial y el proceso de mezcla}
\subsection{Reconstrucción de tuplas}
\subsection{Tipos de joins}
{\itshape Nota: No estoy muy segura de si está bien esto}

Kinética conectar datos relacionados entre dos o más tablas a través de join views. Los tipos de join permitidos son:
\begin{itemize}
  \item INNER
  \item LEFT
  \item RIGHT
  \item FULL OUTER
  \item Cross
\end{itemize}
Existen dos tipos de ejecución para los joins:
\begin{itemize}
  \item Native joins: Realiza un aislamiento de la operación join y crea una join view nativa de Kinética. 
  \item SQL Joins: Se realiza de manera automática cuando una base de datos recibe una query de SQL con la cláusula JOIN. 
\end{itemize}

\subsection{Logging/Recovery}
Kinética cuenta con un registro el cual almacena información como: las interacciones de la base de datos, startup/shutdown, información de errores y más. Este registro puede ser configurado para que se almacene otra información que no está por default.

\subsection{Respaldos}
{\itshape Nota: No está completo pero YO me encargo de terminarlo}

Existe una herramienta para la administración, configuración e instalación de Kinética llamada KAgent, la cual también permite simplificar el proceso de respaldo y 

\subsection{Manejo de transacciones}
\subsection{Cold/hot store}
\subsection{Manejo de datos históricos}

\end{document}
\endinput
%%
%% End of file `sample-acmsmall.tex'.
